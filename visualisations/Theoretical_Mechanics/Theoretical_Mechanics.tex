\documentclass{book}
\usepackage{float}
\usepackage{graphicx}
\usepackage{amsmath}
\usepackage{amssymb}
\title{Figures Theoretical Mechanics}
\author{Felix Claeys, Brecht Verbeken, Simon Verbruggen}
\begin{document}
\maketitle
\section*{4.4 Euler angles}
\begin{figure}[H]
\centering
\includegraphics{CH4/4.4_Euler_angles/standalone.pdf}
\caption{The rotations defining the Eulerian angles.}
\end{figure}

\section*{4.8 Infinitesimal rotations}
\begin{figure}[H]
\centering
\includegraphics{CH4/4.8_Infinitesimal_rotations/standalone.pdf}
\caption{Change in a vector $\boldsymbol{r}$ produced by an infinitesimal clockwise rotation $\mathrm{d}\Phi$ of the vector.}
\end{figure}

\subsection*{4.10.1 coriolis on sphere}
\begin{figure}[H]
\centering
\includegraphics{CH4/4.10_The_Coriolis_effect/4.10.1_coriolis_on_sphere/standalone.pdf}
\caption{}
\end{figure}

\subsection*{4.10.2 coriolis effect local}
\begin{figure}[H]
\centering
\includegraphics{CH4/4.10_The_Coriolis_effect/4.10.2_coriolis_effect_local/standalone.pdf}
\caption{}
\end{figure}

\section*{5.1 slide 53 proof homogeneous pointsymmetric bodies(underconstruction)}
\begin{figure}[H]
\centering
\includegraphics{CH5/5.1_slide_53_proof_homogeneous_pointsymmetric_bodies(underconstruction)/standalone.pdf}
\caption{}
\end{figure}

\section*{5.2 traagheidstensor balk 2D visualisatie}
\begin{figure}[H]
\centering
\includegraphics{CH5/5.2_traagheidstensor_balk_2D_visualisatie/standalone.pdf}
\caption{Graphical representation of the relationship between the angular velocity vector $\bm{\omega}$ (gray) and the angular momentum vector $\bm{L}$ (red) for a rectangular beam with inertia tensor $\mathbf{I}$. 
The directions of $\bm{L}$ and $\bm{\omega}$ do not coincide due to the anisotropy of the inertia tensor. 
The rectangular beam has dimensions 2, 1, and 0.2 along the x, y, and z axes respectively and is homogeneous in density.}
\end{figure}

\section*{5.3 traagheidstensor ellipsoïde 2D visualisatie}
\begin{figure}[H]
\centering
\includegraphics{CH5/5.3_traagheidstensor_ellipsoïde_2D_visualisatie/standalone.pdf}
\caption{Graphical representation of the relationship between the angular velocity vector $\bm{\omega}$ (gray) and the angular momentum vector $\bm{L}$ (red) for an ellipsoid with inertia tensor $\mathbf{I}$. 
The directions of $\bm{L}$ and $\bm{\omega}$ do not coincide due to the anisotropy of the inertia tensor. 
The ellipsoid has dimensions 3, 1.5 and 1.5 along the x, y, and z axes respectively and is homogeneous in density.}
\end{figure}

\section*{8.1 Afbeelding1}
\begin{figure}[H]
\centering
\includegraphics{CH8/8.1_chap/8.1_Afbeelding1/standalone.pdf}
\caption{De krachten op een deeltje door een elektrisch en magnetisch veld
}
\end{figure}

\section*{8.2 Slide 187}
\begin{figure}[H]
\centering
\includegraphics{CH8/8.1_chap/8.2_Slide_187/standalone.pdf}
\caption{De scalaire potentiaal legt een elektrisch veld op}
\end{figure}

\section*{8.3 Slide 187b}
\begin{figure}[H]
\centering
\includegraphics{CH8/8.1_chap/8.3_Slide_187b/standalone.pdf}
\caption{De vectorpotentiaal en het magnetisch veld}
\end{figure}

\end{document}