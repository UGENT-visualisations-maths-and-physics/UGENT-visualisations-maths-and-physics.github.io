\documentclass{standalone}
\usepackage{tikz}
\usepackage{float}
\usepackage{amsmath}
\usepackage{lmodern}
\usepackage{amssymb}
\usepackage{pgfplots}
\usetikzlibrary{calc}
\usetikzlibrary{hobby}
\usepackage{nicefrac}
\usetikzlibrary{decorations.markings}
\usetikzlibrary{patterns, patterns.meta}
\usetikzlibrary{shapes}
\usetikzlibrary{shapes.misc}
\pgfplotsset{compat=1.18}
\begin{document}
\centering

\begin{tikzpicture}
    \tikzset{arrowstyle/.style={->, >=stealth},
            Pole/.style={solid, cross out, minimum size=4pt, inner sep=0pt,draw=black}  % define style for poles=cross
            }

    % define sizes of the circles
    \pgfmathsetmacro{\DomainCircleRadius}{2.0}    % Radius of the circle in the domain
    % artificially increases second circle size for clarity
     \pgfmathsetmacro{\CircleRadiusScale}{3.0}
     \pgfmathsetmacro{\RangeCircleRadius}{(1/\DomainCircleRadius)*\CircleRadiusScale}

    %% MACRO'S

    \newcommand{\DrawPole}[2]{%
    %function for drawing poles
    % #Inputs
    % #1 = coordinate (can be polar like 71.6:1.58)
    % #2 = label (e.g., $z_1$)
    \node[Pole] at (#1) {};
    \node[below right] at (#1) {#2};
    }

    \newcommand{\InverseCoord}[2]{%
    % Inverse a coordinate w=1/z
    % Parse inputs
    \pgfmathsetmacro{\Angle}{#1}
    \pgfmathsetmacro{\Radius}{#2}
    % perform inversion
    \pgfmathsetmacro{\InverseAngle}{-(\Angle)}%
    \pgfmathsetmacro{\InverseRadius}{1/(\Radius) * \CircleRadiusScale}%
    }

    \colorlet{BlueBackground}{blue!5}
    % Background for entire canvas
    \fill[BlueBackground!5] (-8,-4) rectangle (8,4);

    % === Figure positions ===
    % We'll define each figure in its own scope with a shift

    % arrow between the curves

    \draw[thick,postaction={decorate}, decoration={markings,mark=at position 0.5 with {\arrow{stealth}}}][black] (-1.5,0.4)
    .. controls (-1,0.9) and (1,0.9) .. (1.5,0.4);
    \node[above] at (0,1.2) {$w=\frac{1}{z} $};

    
    % ----------------------
   
    % define the POLES
    % pole 1
    \pgfmathsetmacro{\PoleRadiusA}{1.58}
    \pgfmathsetmacro{\PoleAngleA}{71.6}
     % pole 2
    \pgfmathsetmacro{\PoleRadiusB}{1.48}
    \pgfmathsetmacro{\PoleAngleB}{28.8}
    % pole 3
    \pgfmathsetmacro{\PoleRadiusC}{1.13}
    \pgfmathsetmacro{\PoleAngleC}{-45}
    % pole 4
    \pgfmathsetmacro{\PoleRadiusD}{1.08}
    \pgfmathsetmacro{\PoleAngleD}{146.3}

    % First Figure
    \begin{scope}[shift={(-4.5,0)}]
    % Assen
    \draw[thick, arrowstyle] ($(-3,0)$) -- ($(3,0)$);
    \draw[thick, arrowstyle] ($(0,-3)$) -- ($(0,3)$);
    \coordinate (x) at (3,0);
    \draw (x) node[below right] {$x$};
    \coordinate (y) at (0,3);
    \draw (y) node[below right] {$y$};
    % Cirkel
    
    \draw[semithick, postaction={decorate}, decoration={markings,mark=at position 0.125 with {\arrow{stealth}},mark=at position 0.375 with {\arrow{stealth}},mark=at position 0.625 with {\arrow{stealth}},mark=at position 0.875 with {\arrow{stealth}}}] (\DomainCircleRadius,0) arc[start angle=0, end angle=360, radius=\DomainCircleRadius];
    \coordinate (G) at (1.6,1.6);
    \draw (G) node[below right] {$\gamma$};

    %draw poles
    \DrawPole{51.9:3.562}{$\infty$};

    % draw the poles inside the circle
    \DrawPole{\PoleAngleA:\PoleRadiusA}{$z_1$}
    \DrawPole{\PoleAngleB:\PoleRadiusB}{$z_2$}
    \DrawPole{\PoleAngleC:\PoleRadiusC}{$z_3$}
    \DrawPole{\PoleAngleD:\PoleRadiusD}{$z_4$}

    \end{scope}

    
    \begin{scope}[shift={(4.5,0)}]
    % Assen
    \draw[thick, arrowstyle] ($(-3,0)$) -- ($(3,0)$);
    \draw[thick, arrowstyle] ($(0,-3)$) -- ($(0,3)$);
    \coordinate (u) at (3,0);
    \draw (u) node[below right] {$u$};
    \coordinate (v) at (0,3);
    \draw (v) node[below right] {$v$};
    % Cirkel

    \draw[semithick, postaction={decorate}, decoration={markings,mark=at position 0.125 with {\arrow{stealth}},mark=at position 0.375 with {\arrow{stealth}},mark=at position 0.625 with {\arrow{stealth}},mark=at position 0.875 with {\arrow{stealth}}}] (0,0) circle[start angle=0, end angle=-360, radius=\RangeCircleRadius];
    \coordinate (W) at (-135:\RangeCircleRadius*1.15);
    \draw (W) node[below right] {$\gamma^-$};


    %draw poles: first calculate inverse point via the macro \InverseCoord
    \InverseCoord{\PoleAngleA}{\PoleRadiusA}
    \DrawPole{\InverseAngle:\InverseRadius}{$w_1$}
    \InverseCoord{\PoleAngleB}{\PoleRadiusB}
    \DrawPole{\InverseAngle:\InverseRadius}{$w_2$}
    \InverseCoord{\PoleAngleC}{\PoleRadiusC}
    \DrawPole{\InverseAngle:\InverseRadius}{$w_3$}
    \InverseCoord{\PoleAngleD}{\PoleRadiusD}
    \DrawPole{\InverseAngle:\InverseRadius}{$w_4$}

    % pole at infinity now ot origin
    \DrawPole{0:0}{$w_{\infty}$};
    \end{scope}
\end{tikzpicture}
\end{document}