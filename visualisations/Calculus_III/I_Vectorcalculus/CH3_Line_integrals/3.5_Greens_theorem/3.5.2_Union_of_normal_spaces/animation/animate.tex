\documentclass{standalone}
\usepackage{tikz}
\usepackage{animate}
\usepackage{float}
\usepackage{amsmath}
\usepackage{lmodern}
\usepackage{amssymb}
\usepackage{ifthen}
\usetikzlibrary{calc}
\usetikzlibrary{intersections}
\usetikzlibrary{decorations.markings}
\usetikzlibrary{patterns, patterns.meta}
\usetikzlibrary{shapes}

\begin{document}

\centering

% Put all non-frame dependend code in front (faster compilation and ensures uniform figures)
\tikzset{
    BigTextFont/.style={font=\large},
    every node/.style={font=\normalsize, text=black},
    CircleNodeStyle/.style={draw=black, shape=circle, fill=black, minimum size=\CircleSize*2 cm, inner sep=0pt},
    arrowstyle/.style={->, >=stealth}}
\pgfmathsetmacro{\CircleSize}{0.08} 
\pgfmathsetmacro{\InnerCircleRadius}{1.5}
\pgfmathsetmacro{\OuterCircleRadius}{3.0}
\pgfmathsetmacro{\MaxSize}{5.0}


\pgfmathsetmacro{\fps}{20}   % frames per second

% first series of frames: gradually draw partition lines
% first series of frames variables
\pgfmathsetmacro{\FramesA}{20}                          % total amount of frames in this series
\pgfmathsetmacro{\IncrementSeriesA}{1.0 / \FramesA}     % increment that increases the factor that lines are drawn per frame

% second series of frames: gradually split the partitions apart
% second series of frames variables
\pgfmathsetmacro{\FramesB}{30}
\pgfmathsetmacro{\SplitPartMax}{1.5}                     % Max separation of the 4 partitions at the end of the animation
\pgfmathsetmacro{\IncrementSeriesB}{1.5 / \FramesB}      % increment that increases separation of the 4 partitions

% third series of frames: the labels of the 4 areas are gradually displayed
% third series of frames variables
\pgfmathsetmacro{\FramesC}{5}

% fourth series of frames: gradually show arrows around contour. 
% fourth series of frames: variables
\pgfmathsetmacro{\FramesD}{6}
\pgfmathsetmacro{\FramesPerArrowA}{1}           % determines after which frame the first arrow is displayed
\pgfmathsetmacro{\FramesPerArrowBA}{2}          % determines after which frame the second arrow is displayed
\pgfmathsetmacro{\FramesPerArrowBB}{3}          % etc.
\pgfmathsetmacro{\FramesPerArrowC}{4}
\pgfmathsetmacro{\FramesPerArrowD}{5}

\begin{animateinline}{\fps} % start animation and set fps
    % first series of frames
    \multiframe{\FramesA}{r=0+\IncrementSeriesA}{ %\FramesA are the amount of frames, r is the factor of the lines that are drawn starts from zero and each frame an increment is added

        \begin{tikzpicture}[scale=0.8]
        % same bounding box must be used over all frames to ensure same frame size!
        \useasboundingbox (-\MaxSize, -\MaxSize) rectangle (\MaxSize, \MaxSize);

        \pgfmathsetmacro{\FractionDraw}{\r}     % define \r in Tikz environment with a better variable name
        % draw  the area
        \draw[very thick] (0,0) circle [radius=\InnerCircleRadius];
        \draw[very thick] (0,0) circle [radius=\OuterCircleRadius];

        % define the endpoints of the partially drawn lines
        \coordinate (UpperEndPoint) at ($(0,\InnerCircleRadius)!\FractionDraw!(0,\OuterCircleRadius)$);
        \coordinate (LowerEndPoint) at ($(0,-\InnerCircleRadius)!\FractionDraw!(0,-\OuterCircleRadius)$);
        \coordinate (RightEndPoint) at ($(\InnerCircleRadius, 0)!\FractionDraw!(\OuterCircleRadius, 0)$);
        \coordinate (LeftEndPoint) at ($(-\InnerCircleRadius, 0)!\FractionDraw!(-\OuterCircleRadius, 0)$);

        % partitions
        \draw[thick] (0,\InnerCircleRadius) -- (UpperEndPoint);
        \draw[thick] (0,-\InnerCircleRadius) -- (LowerEndPoint);
        \draw[thick] (\InnerCircleRadius, 0) -- (RightEndPoint);
        \draw[thick] (-\InnerCircleRadius, 0) -- (LeftEndPoint);

        % fill area with pattern (even odd rule for exclusion of inner circle)
        \fill[even odd rule, pattern={Lines[angle=30, distance=0.4cm, line width=0.6pt]}, pattern color=gray] circle (\InnerCircleRadius) circle (\OuterCircleRadius);

        % G area symbol
        \pgfmathsetmacro{\NodeRadius}{(\InnerCircleRadius + \OuterCircleRadius)/2}
        \node[draw=white, circle, inner sep=0, fill=white, BigTextFont] at (45 : \NodeRadius) {G};

    \end{tikzpicture}
    }

    \newframe         % newframe is needed for smooth transition between frames!
    % second series of frames
    \multiframe{\FramesB}{r=\IncrementSeriesB+\IncrementSeriesB}{ % \PartSplit is the distance that the partitions are split apart
    
        \begin{tikzpicture}[scale=0.8]
        % same bounding box must be used over all frames to ensure same frame size
        \useasboundingbox (-\MaxSize, -\MaxSize) rectangle (\MaxSize, \MaxSize);
        
        \pgfmathsetmacro{\PartSplit}{\r}        
        
        \foreach \StartAngle/\EndAngle/\label in {0/90/{$D_{3}$}, 90/180/{$D_{4}$}, 180/270/{$D_{1}$}, 270/360/{$D_{2}$}} {

        % shift each part in a different direction
        \pgfmathsetmacro{\xshift}{\PartSplit*cos((\StartAngle+\EndAngle)/2)}
        \pgfmathsetmacro{\yshift}{\PartSplit*sin((\StartAngle+\EndAngle)/2)}
        \begin{scope}[shift={(\xshift, \yshift)}]
            % draw the 4 edges of the part, with arrows counterclockwise
            \draw[very thick] (\StartAngle :\InnerCircleRadius) arc [start angle=\StartAngle, end angle=\EndAngle, radius=\InnerCircleRadius];
            \draw[very thick] (\StartAngle :\OuterCircleRadius) arc [start angle=\StartAngle, end angle=\EndAngle, radius=\OuterCircleRadius];
            
            \draw[very thick] (\StartAngle :\InnerCircleRadius) -- (\StartAngle :\OuterCircleRadius);

            \draw[very thick] (\EndAngle :\InnerCircleRadius) -- (\EndAngle :\OuterCircleRadius);
             % fill the contour
            \fill[pattern={Lines[angle=30, distance=0.4cm, line width=0.6pt]}, pattern color=gray] (\EndAngle :\InnerCircleRadius) arc [start angle=\EndAngle, end angle=\StartAngle, radius=\InnerCircleRadius] -- (\StartAngle :\InnerCircleRadius) -- (\StartAngle :\OuterCircleRadius) -- (\StartAngle :\OuterCircleRadius) arc [start angle=\StartAngle, end angle=\EndAngle, radius=\OuterCircleRadius] -- cycle;
        \end{scope}
        }
    \end{tikzpicture}
    }

    % third series of frames:
    \newframe*[3]       % newframe* makes the animation wait for a mouseclick before continuing. We also set the framerate to 3 here
    \multiframe{\FramesC}{r=1.0+1.0}{ % r is used as index and using an if statement, the labels are displayed

        \begin{tikzpicture}[scale=0.8]
        % same bounding box must be used over all frames to ensure same frame size
        \useasboundingbox (-\MaxSize, -\MaxSize) rectangle (\MaxSize, \MaxSize);
        
        \pgfmathsetmacro{\PartSplit}{\SplitPartMax}

        % include extra parameter to extract index
        \foreach \Index/\StartAngle/\EndAngle/\label in {3/0/90/{$D_{3}$}, 4/90/180/{$D_{4}$}, 1/180/270/{$D_{1}$}, 2/270/360/{$D_{2}$}} {

        % shift each part in a different direction
        \pgfmathsetmacro{\xshift}{\PartSplit*cos((\StartAngle+\EndAngle)/2)}
        \pgfmathsetmacro{\yshift}{\PartSplit*sin((\StartAngle+\EndAngle)/2)}
        \begin{scope}[shift={(\xshift, \yshift)}]
            % draw the 4 edges of the part, with arrows counterclockwise
            \draw[very thick] (\StartAngle :\InnerCircleRadius) arc [start angle=\StartAngle, end angle=\EndAngle, radius=\InnerCircleRadius];

            \draw[very thick] (\StartAngle :\InnerCircleRadius) -- (\StartAngle :\OuterCircleRadius);

            \draw[very thick] (\StartAngle :\OuterCircleRadius) arc [start angle=\StartAngle, end angle=\EndAngle, radius=\OuterCircleRadius];

            \draw[very thick] (\EndAngle :\InnerCircleRadius) -- (\EndAngle :\OuterCircleRadius);


            % fill the contour
            \fill[pattern={Lines[angle=30, distance=0.4cm, line width=0.6pt]}, pattern color=gray] (\EndAngle :\InnerCircleRadius) arc [start angle=\EndAngle, end angle=\StartAngle, radius=\InnerCircleRadius] -- (\StartAngle :\InnerCircleRadius) -- (\StartAngle :\OuterCircleRadius) -- (\StartAngle :\OuterCircleRadius) arc [start angle=\StartAngle, end angle=\EndAngle, radius=\OuterCircleRadius] -- cycle;

            \pgfmathsetmacro{\IndexNormalised}{\Index * \FramesC/ (4*1.4)}
            \pgfmathsetmacro{\NodeRadius}{(\InnerCircleRadius + \OuterCircleRadius)/2}
            \pgfmathsetmacro{\NodeAngle}{(\StartAngle + \EndAngle)/2}

            % if statement to display label increasingly with index
            % if num doesn't accept non integers! (-> use pgfmathparse)
            \pgfmathparse{\r > \IndexNormalised}
            \ifnum \pgfmathresult=1
            \node[draw=white, circle, inner sep=0, fill=white, BigTextFont] at (\NodeAngle : \NodeRadius) {\label};
            \fi

        \end{scope}
        }
    \end{tikzpicture}
    }

    % fourth series of frames:
    % The code uses a lot of if statements because the contours are split in 4 different components.
    % for arrows around a single contour something along the lines of "mark = between positions 0.0 and \r step 0.1 with {\arrow{stealth}}" would work nicely
    \newframe   % is needed for smooth transition
    \multiframe{\FramesD}{r=1.0+1.0}{ % 20 frames, increasing \PartSplit from 0.1 in steps of 0.05

        \begin{tikzpicture}[scale=0.8]
        % same bounding box must be used over all frames to ensure same frame size
        \useasboundingbox (-\MaxSize, -\MaxSize) rectangle (\MaxSize, \MaxSize);
        
        \pgfmathsetmacro{\PartSplit}{\SplitPartMax}
        
        \foreach \StartAngle/\EndAngle/\label in {0/90/{$D_{3}$}, 90/180/{$D_{4}$}, 180/270/{$D_{1}$}, 270/360/{$D_{2}$}} {

        % shift each part in a different direction
        \pgfmathsetmacro{\xshift}{\PartSplit*cos((\StartAngle+\EndAngle)/2)}
        \pgfmathsetmacro{\yshift}{\PartSplit*sin((\StartAngle+\EndAngle)/2)}
        \begin{scope}[shift={(\xshift, \yshift)}]
            % draw the 4 edges of the part, with arrows counterclockwise
            \ifnum \r > \FramesPerArrowA
            \draw[postaction={decorate}, decoration={
                markings,
                mark=at position 0.50 with {\arrow{stealth}}}]
            [very thick] (\StartAngle :\InnerCircleRadius) -- (\StartAngle :\OuterCircleRadius);
            \else
            \draw[very thick] (\StartAngle :\InnerCircleRadius) -- (\StartAngle :\OuterCircleRadius);
            \fi
            
            \ifnum \r > \FramesPerArrowBB
            \draw[postaction={decorate}, decoration={
                markings,
                mark = between positions 0.3 and 1 step 0.375 with {\arrow{stealth}}}]
            [very thick] (\StartAngle :\OuterCircleRadius) arc [start angle=\StartAngle, end angle=\EndAngle, radius=\OuterCircleRadius];
            \else
            \draw[very thick] (\StartAngle :\OuterCircleRadius) arc [start angle=\StartAngle, end angle=\EndAngle, radius=\OuterCircleRadius];  
            \fi

            \ifnum \r > \FramesPerArrowBA
            \draw[postaction={decorate}, decoration={
                markings,
                mark = at position 0.30 with {\arrow{stealth}}}]
            [very thick] (\StartAngle :\OuterCircleRadius) arc [start angle=\StartAngle, end angle=\EndAngle, radius=\OuterCircleRadius];
            \fi

            \ifnum \r > \FramesPerArrowC
            \draw[postaction={decorate}, decoration={
                markings,
                mark=at position 0.50 with  {\arrowreversed{stealth}}}]
            [very thick] (\EndAngle :\InnerCircleRadius) -- (\EndAngle :\OuterCircleRadius);
            \else
            \draw[very thick] (\EndAngle :\InnerCircleRadius) -- (\EndAngle :\OuterCircleRadius);
            \fi

            \ifnum \r > \FramesPerArrowD
            \draw[postaction={decorate}, decoration={
            markings,
            mark = between positions 0.375 and 1 step 0.75 with {\arrowreversed{stealth}}}]
            [very thick] (\StartAngle :\InnerCircleRadius) arc [start angle=\StartAngle, end angle=\EndAngle, radius=\InnerCircleRadius];
            \else
            \draw[very thick] (\StartAngle :\InnerCircleRadius) arc [start angle=\StartAngle, end angle=\EndAngle, radius=\InnerCircleRadius];
            \fi

            % fill the contour
            \fill[pattern={Lines[angle=30, distance=0.4cm, line width=0.6pt]}, pattern color=gray] (\EndAngle :\InnerCircleRadius) arc [start angle=\EndAngle, end angle=\StartAngle, radius=\InnerCircleRadius] -- (\StartAngle :\InnerCircleRadius) -- (\StartAngle :\OuterCircleRadius) -- (\StartAngle :\OuterCircleRadius) arc [start angle=\StartAngle, end angle=\EndAngle, radius=\OuterCircleRadius] -- cycle;

            \pgfmathsetmacro{\NodeRadius}{(\InnerCircleRadius + \OuterCircleRadius)/2}
            \pgfmathsetmacro{\NodeAngle}{(\StartAngle + \EndAngle)/2}
            % display label 
            \node[draw=white, circle, inner sep=0, fill=white, BigTextFont] at (\NodeAngle : \NodeRadius) {\label};
            
        \end{scope}
        }
    \end{tikzpicture}
    }
\end{animateinline}

\end{document}