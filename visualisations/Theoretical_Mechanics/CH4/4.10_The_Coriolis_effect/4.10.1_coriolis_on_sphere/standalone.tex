\documentclass{standalone}
\usepackage{pgfplots}
\usetikzlibrary{calc}
\usepackage{xxcolor}
\usepackage{amsmath}
\usepackage{bm}
\pgfplotsset{compat=1.10}
% used code for detection of points on the front and the back of an object (PGF) (this is very useful!) (https://tex.stackexchange.com/questions/199161/parabolic-moebius-map-on-sphere-using-tikz/199715#199715)

% Declare nice sphere shading: http://tex.stackexchange.com/a/54239/12440
\pgfdeclareradialshading[tikz@ball]{ball}{\pgfqpoint{0bp}{0bp}}{%
 color(0bp)=(tikz@ball!0!white);
 color(7bp)=(tikz@ball!0!white);
 color(15bp)=(tikz@ball!70!black);
 color(20bp)=(black!70);
 color(30bp)=(black!70)}
\makeatother

\pgfmathsetmacro{\SphereRadius}{4.0}
\pgfmathsetmacro{\OpacityInside}{0.25}

% Style to set TikZ camera angle, like PGFPlots `view`
\tikzset{viewport/.style 2 args={
    x={({cos(-#1)*\SphereRadius cm},{sin(-#1)*sin(#2)*\SphereRadius cm})},
    y={({-sin(-#1)*\SphereRadius cm},{cos(-#1)*sin(#2)*\SphereRadius cm})},
    z={(0,{cos(#2)*\SphereRadius cm})}
}}

% Styles to plot only points that are before or behind the sphere.
\pgfplotsset{only foreground/.style={
    restrict expr to domain={rawx*\CameraX + rawy*\CameraY + rawz*\CameraZ}{-0.05:100},
}}
\pgfplotsset{only background/.style={
    restrict expr to domain={rawx*\CameraX + rawy*\CameraY + rawz*\CameraZ}{-100:0.05}
}}

% Automatically plot transparent lines in background and solid lines in foreground
\def\addFGBGplot[#1]#2;{
    \addplot3[#1,only background, opacity=\OpacityInside] #2;
    \addplot3[#1,only foreground] #2;
}

\colorlet{Green}{black!60!green}

\tikzset{every node/.style={font=\normalsize,text=black},
        arrowstyle/.style={->, >=stealth}}

\newcommand{\ViewAzimuth}{15}
\newcommand{\ViewElevation}{25}

\begin{document}
\begin{tikzpicture}
    % Compute camera unit vector for calculating depth
    \pgfmathsetmacro{\CameraX}{sin(\ViewAzimuth)*cos(\ViewElevation)}
    \pgfmathsetmacro{\CameraY}{-cos(\ViewAzimuth)*cos(\ViewElevation)}
    \pgfmathsetmacro{\CameraZ}{sin(\ViewElevation)}
    \path[use as bounding box] (-5,-5) rectangle (5,5); % Avoid jittering animation
    % Draw a nice looking sphere
    \begin{scope}
        \clip (0,0) circle (\SphereRadius);
        \begin{scope}[scale = \SphereRadius, transform canvas={rotate=-20}]
            \shade [ball color=white] (0,0.5) ellipse (1.8 and 1.5);
        \end{scope}
    \end{scope}
    \begin{axis}[
        hide axis,
        view={\ViewAzimuth}{\ViewElevation},     % Set view angle
        every axis plot/.style={semithick},
        disabledatascaling,                      % Align PGFPlots coordinates with TikZ
        anchor=origin,                           % Align PGFPlots coordinates with TikZ
        viewport={\ViewAzimuth}{\ViewElevation}, % Align PGFPlots coordinates with TikZ
    ]
        % Plot equator and two longitude lines with occlusion
        \addFGBGplot[domain=0:2*pi, samples=100, samples y=1, very thick] ({cos(deg(x))}, {sin(deg(x))}, 0);
        \addFGBGplot[domain=0:2*pi, samples=100, samples y=1] (0, {sin(deg(x))}, {cos(deg(x))});
        \addFGBGplot[domain=0:2*pi, samples=100, samples y=1] ({sin(deg(x))}, 0, {cos(deg(x))});

        % some background elements
        % draw earth's rotation
        \addFGBGplot[domain=-pi*8/9:-pi*11/9, samples=100, samples y=1, line width=2.3pt, ->] ({1.2*sin(deg(x))}, {1.2*cos(deg(x))}, {0});
        % draw omega
        \begin{scope}[gray!60!black, ultra thick]
            \draw[->, >=latex] (0, 0, 1) -- (0, 0, 1.3) node[pos=1.0, left] {$\pmb{\omega}$};
        \end{scope}

        % point considered on the sphere
        \pgfmathsetmacro{\ThetaA}{60}
        \pgfmathsetmacro{\PhiA}{270}
        \pgfmathsetmacro{\ThetaARad}{\ThetaA*(pi/180)}
        \pgfmathsetmacro{\PhiARad}{\PhiA*(pi/180)}
        \pgfmathsetmacro{\dTheta}{-pi/5}

        %% draw tangent plane through point on sphere
        \newcommand{\sphericalcoord}[3]{ 
        \coordinate (TempPoint) at ({#1*sin(#2)*cos(#3)}, {#1*sin(#2)*sin(#3)}, {#1*cos(#2)});
        }

        % define the start point
        \sphericalcoord{1.0}{\ThetaA}{\PhiA}
        \coordinate (BeginPoint) at (TempPoint);

        \def\a{sin(\ThetaA)*cos(\PhiA)}
        \def\b{sin(\ThetaA)*sin(\PhiA)}
        \def\c{cos(\ThetaA)}
        \def\Const{\a*\a + \b*\b + \c*\c}      % constant in top of equation: a*x0+b*y0+c*z0

        % function for calculating r of point on a tangent plane for given theta, phi
        \pgfmathdeclarefunction{rplane}{2}{%
        % Arguments: #1=theta, #2=phi
        \pgfmathparse{(\Const)/(\a*sin(#1)*cos(#2) + \b*sin(#1)*sin(#2) + \c*cos(#1))}%
        }

        % define dimensions of plane, in terms of spherical angles
        \pgfmathsetmacro{\ThetaTop}{\ThetaA-21}
        \pgfmathsetmacro{\ThetaBottom}{\ThetaA+7}
        \pgfmathsetmacro{\PhiOpening}{\PhiA-22}
        \pgfmathsetmacro{\PhiMultRight}{2.6}

        % calculate curve in the middle 
        \pgfmathsetmacro{\rA}{rplane(\ThetaTop, \PhiA)}
        \sphericalcoord{\rA}{\ThetaTop}{\PhiA}
        \coordinate (Topcurve) at (TempPoint);
        \pgfmathsetmacro{\rA}{rplane(\ThetaBottom, \PhiA)}
        \sphericalcoord{\rA}{\ThetaBottom}{\PhiA}
        \coordinate (BottomCurve) at (TempPoint);
        \coordinate (VerticalAngle) at ($(Topcurve) - (BottomCurve)$);

        % calculate the coordinates of the plane's corner points
        \pgfmathsetmacro{\rA}{rplane(\ThetaTop, \PhiOpening)}
        \sphericalcoord{\rA}{\ThetaTop}{\PhiOpening}
        \coordinate (TopLeftCorner) at (TempPoint);
        % calculate horizontal distance
        \coordinate (HalfWidthPlane) at ($(Topcurve) - (TopLeftCorner)$);
        % construct the remaining corners based on the incline of the plane and 
        \coordinate (BottomLeftCorner) at ($(TopLeftCorner) - (VerticalAngle)$);
        \coordinate (BottomRightCorner) at ($(BottomLeftCorner) + \PhiMultRight*(HalfWidthPlane)$);
        \coordinate (TopRightCorner) at ($(TopLeftCorner) + \PhiMultRight*(HalfWidthPlane)$);

        % draw part of arrows behind plane at lower opacity
        \begin{scope}[very thick]
            \clip (TopRightCorner) -- (TopLeftCorner) -- (BottomLeftCorner) -- (BottomRightCorner) -- cycle;
            % draw unimpeded path
            \addFGBGplot[domain=\ThetaARad:\ThetaARad+\dTheta, samples=100, samples y=1, arrowstyle, Green!60] ({sin(deg(x))*cos(\PhiA)}, {sin(deg(x))*sin(\PhiA)}, {cos(deg(x))});
            % draw deflected path
            \def\phi{-deg((6.5*sin(deg(0.20*(x-\ThetaARad))))^3) + \PhiA}
            \def\theta{deg(x)}
            \addFGBGplot[domain=\ThetaARad:\ThetaARad+0.9*\dTheta, samples=100, samples y=1, arrowstyle, blue!60] ({sin(\theta)*cos(\phi)}, {sin(\theta)*sin(\phi)}, {cos(\theta)});
        \end{scope}
        
        % draw other parts or arrows normally
        \begin{scope}[very thick]
            \clip (TopRightCorner) -- (TopLeftCorner) -- (BottomLeftCorner) -- (BottomRightCorner) -- cycle
                (-3,-3) rectangle (3,3);
            \addFGBGplot[domain=\ThetaARad:\ThetaARad+\dTheta, samples=100, samples y=1, arrowstyle, Green] ({sin(deg(x))*cos(\PhiA)}, {sin(deg(x))*sin(\PhiA)}, {cos(deg(x))});
            % draw deflected path
            \def\phi{-deg((6.5*sin(deg(0.20*(x-\ThetaARad))))^3) + \PhiA}
            \def\theta{deg(x)}
            \addFGBGplot[domain=\ThetaARad:\ThetaARad+0.9*\dTheta, samples=100, samples y=1, arrowstyle, blue!60] ({sin(\theta)*cos(\phi)}, {sin(\theta)*sin(\phi)}, {cos(\theta)});
        \end{scope}

        % draw the tangent plane
        \fill[red, opacity=0.5] (TopRightCorner) -- (TopLeftCorner) -- (BottomLeftCorner) -- (BottomRightCorner) -- cycle;
        \draw[very thick] (TopRightCorner) -- (TopLeftCorner) -- (BottomLeftCorner) -- (BottomRightCorner) -- cycle;
        % draw the velocities on the plane
        \draw[arrowstyle, thick] (BeginPoint) -- (Topcurve) node[pos=0.5, left] {$\mathbf{v}_0$};
        \pgfmathsetmacro{\PhiMultRightRest}{\PhiMultRight-1.1}
        \draw[arrowstyle, thick] (BeginPoint) -- ($(BeginPoint)+\PhiMultRightRest*(HalfWidthPlane)$) node[pos=0.4, below] {$\mathbf{v} \times \pmb{\omega}$};
        
        % indicate the starting coordinate
        \fill[white] (BeginPoint) circle (2pt);
        \draw[semithick] (BeginPoint) circle (2pt);
        % draw the normal
        \draw[opacity=\OpacityInside] (0, 0, 0) -- (BeginPoint);
        \draw[arrowstyle, thick] (BeginPoint) -- ($1.5*(BeginPoint)$) node[pos=1.0, left] {$\pmb{\omega}_z$};
    

    \end{axis}
\end{tikzpicture}
\end{document}