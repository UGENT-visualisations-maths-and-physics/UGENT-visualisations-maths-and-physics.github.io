\documentclass{book}
\usepackage{float}
\usepackage{graphicx}
\usepackage{amsmath}
\usepackage{amssymb}
\usepackage{bm}
\title{Figures Theoretical Mechanics}
\author{Felix Claeys, Brecht Verbeken, Simon Verbruggen}
\begin{document}
\maketitle
\section*{1.2 Mechanics of Multiple Particles}
\begin{figure}[H]
\centering
\includegraphics{CH1/1.2_Mechanics_of_Multiple_Particles/standalone.pdf}
\caption{Vector notations for points of mass and the center of mass.}
\end{figure}

\section*{1.3 Bindings}
\begin{figure}[H]
\centering
\includegraphics{CH1/1.3_Bindings/standalone.pdf}
\caption{Double pendulum.}
\end{figure}

\section*{1.6 Atwoods Machine}
\begin{figure}[H]
\centering
\includegraphics{CH1/1.6_Atwoods_Machine/standalone.pdf}
\caption{Atwood's machine.}
\end{figure}

\section*{2.6 Conservation Properties and Symmetry Properties}
\begin{figure}[H]
\centering
\includegraphics{CH2/2.6_Conservation_Properties_and_Symmetry_Properties/standalone.pdf}
\caption{Change of position vector under rotation.}
\end{figure}

\section*{4.4 Euler angles}
\begin{figure}[H]
\centering
\includegraphics{CH4/4.4_Euler_angles/standalone.pdf}
\caption{The rotations defining the Eulerian angles.}
\end{figure}

\section*{4.7 Finite Rotations Detail}
\begin{figure}[H]
\centering
\includegraphics{CH4/4.7_Finite_Rotations_Detail/standalone.pdf}
\caption{Detailimage of the circle on which the rotation takes place.}
\end{figure}

\section*{4.7 Finite Rotations}
\begin{figure}[H]
\centering
\includegraphics{CH4/4.7_Finite_Rotations/standalone.pdf}
\caption{Finite rotation around an axis.}
\end{figure}

\section*{4.8 Infinitesimal rotations}
\begin{figure}[H]
\centering
\includegraphics{CH4/4.8_Infinitesimal_rotations/standalone.pdf}
\caption{Change in a vector $\boldsymbol{r}$ produced by an infinitesimal clockwise rotation $\mathrm{d}\Phi$ of the vector.}
\end{figure}

\section*{4.9 Rate of change of a vector}
\begin{figure}[H]
\centering
\includegraphics{CH4/4.9_Rate_of_change_of_a_vector/standalone.pdf}
\caption{Rate change of a vector $\vec{G}$, as perceived by an observer in the fixed body frame: $d \vec{G}_{\text{body}}$ (right) and in the space frame: $d \vec{G}_{\text{space}} = d \vec{G}_{\text{body}} + \omega \times \vec{G}$ (left). The $z'$ axis is chosen such that it aligns with $\omega$.}
\end{figure}

\subsection*{4.10.1 coriolis on sphere}
\begin{figure}[H]
\centering
\includegraphics{CH4/4.10_The_Coriolis_effect/4.10.1_coriolis_on_sphere/standalone.pdf}
\caption{Coriolis effect for an object moving northward on a rotating sphere. The green trajectory shows the motion for when the Coriolis effect is neglected, while the blue trajectory takes the Coriolis effect into account.}
\end{figure}

\subsection*{4.10.2 coriolis effect local}
\begin{figure}[H]
\centering
\includegraphics{CH4/4.10_The_Coriolis_effect/4.10.2_coriolis_effect_local/standalone.pdf}
\caption{Local view of Coriolis deflection (blue) for an object moving northward.}
\end{figure}

\section*{5.2 traagheidstensor balk 2D visualisatie}
\begin{figure}[H]
\centering
\includegraphics{CH5/5.2_traagheidstensor_balk_2D_visualisatie/standalone.pdf}
\caption{Graphical representation of the relationship between the angular velocity vector $\bm{\omega}$ (gray) and the angular momentum vector $\bm{L}$ (red) for a rectangular beam with inertia tensor $\mathbf{I}$. 
The directions of $\bm{L}$ and $\bm{\omega}$ do not coincide due to the anisotropy of the inertia tensor. 
The rectangular beam has dimensions 2, 1, and 0.2 along the x, y, and z axes respectively and is homogeneous in density. }
\end{figure}

\section*{5.3 visual proof parallel axis theorem for pointsymmetrical bodies}
\begin{figure}[H]
\centering
\includegraphics{CH5/5.3_visual_proof_parallel_axis_theorem_for_pointsymmetrical_bodies/standalone.pdf}
\caption{}
\end{figure}

\section*{5.3 parallelaxistheorem}
\begin{figure}[H]
\centering
\includegraphics{CH5/5.3_parallelaxistheorem/standalone.pdf}
\caption{}
\end{figure}

\section*{5.3 traagheidstensor ellipsoïde 2D visualisatie}
\begin{figure}[H]
\centering
\includegraphics{CH5/5.3_traagheidstensor_ellipsoïde_2D_visualisatie/standalone.pdf}
\caption{Graphical representation of the relationship between the angular velocity vector $\bm{\omega}$ (gray) and the angular momentum vector $\bm{L}$ (red) for an ellipsoid with inertia tensor $\mathbf{I}$. 
The directions of $\bm{L}$ and $\bm{\omega}$ do not coincide due to the anisotropy of the inertia tensor. 
The ellipsoid has dimensions 3, 1.5 and 1.5 along the x, y, and z axes respectively and is homogeneous in density. }
\end{figure}

\section*{5.7 Heavy symmetrical top with one point fixed}
\begin{figure}[H]
\centering
\includegraphics{CH5/5.7_Heavy_symmetrical_top_with_one_point_fixed/standalone.pdf}
\caption{From left to right, monotone precession, spin-reversal precession and non-monotome precession.}
\end{figure}

\section*{6.2 water}
\begin{figure}[H]
\centering
\includegraphics{CH6/6.2_Normal_vibrations_Eigenmodes_in_molecules/6.2_water/standalone.pdf}
\caption{Vibrational modes of H$_2$O: 3N=9 degrees of freedom total. 
  3 vibrational (shown), 3 translational, 3 rotational. 
  Modes: symmetric stretch ($q_1=q_2$), asymmetric stretch ($q_1 \neq q_2$), bending ($\theta$).}
\end{figure}

\section*{6.2 carbondioxide}
\begin{figure}[H]
\centering
\includegraphics{CH6/6.2_Normal_vibrations_Eigenmodes_in_molecules/6.2_carbondioxide/standalone.pdf}
\caption{
  Vibrational modes of CO$_2$: 3N=9 degrees of freedom total. 
  4 vibrational (shown: symmetric and asymmetric stretch, bending in/out-plane), 3 translational, 2 rotational. 
  Bending doubly degenerate (linear molecule). The springs between the C-O bonds and the corresponding generalized coordinate $\theta$ are omitted to avoid overloading the figure.}
\end{figure}

\section*{6.3 Normal Coordinates}
\begin{figure}[H]
\centering
\includegraphics{CH6/6.3_Normal_Coordinates/standalone.pdf}
\caption{Expansion around a small diversion in a minumum up to second order.
}
\end{figure}

\section*{8.1 Afbeelding1}
\begin{figure}[H]
\centering
\includegraphics{CH8/8.1_chap/8.1_Afbeelding1/standalone.pdf}
\caption{De krachten op een deeltje door een elektrisch en magnetisch veld
}
\end{figure}

\section*{8.1 Afbeelding1}
\begin{figure}[H]
\centering
\includegraphics{CH8/8.1_/8.1_Afbeelding1/standalone.pdf}
\caption{}
\end{figure}

\section*{8.1 Slide 187b}
\begin{figure}[H]
\centering
\includegraphics{CH8/8.1_/8.1_Slide_187b/standalone.pdf}
\caption{}
\end{figure}

\section*{8.1 Slide 187}
\begin{figure}[H]
\centering
\includegraphics{CH8/8.1_/8.1_Slide_187/standalone.pdf}
\caption{}
\end{figure}

\section*{8.2 Slide 187}
\begin{figure}[H]
\centering
\includegraphics{CH8/8.1_chap/8.2_Slide_187/standalone.pdf}
\caption{De scalaire potentiaal legt een elektrisch veld op}
\end{figure}

\section*{8.3 Slide 187b}
\begin{figure}[H]
\centering
\includegraphics{CH8/8.1_chap/8.3_Slide_187b/standalone.pdf}
\caption{De vectorpotentiaal en het magnetisch veld}
\end{figure}

\end{document}